\documentclass[../main_frameworkreview.tex]{subfiles}

\begin{document}

\section{openFrameworks}

\subsection{Learning and exploring capabilities}
openFrameworks has an online guide of sorts called ofBook. That said, a lot of the examples in the chapter on sound used deprecated code.\cite{Carlucci} This meant that I had to cross reference the examples with those provided on the Git repository to fully understand what was going on. ofBook is an ongoing project, however, with the writers inviting readers to post issues on the Git repository page and so these guides would ideally improve with time. openFrameworks also has a comprehensive documentation page of their proprietary classes and functions. These proprietary classes can be identified by the \verb|of| prefix --- \verb|ofBaseApp|, for example, provides us with functions that deal with user input.
\subsubsection{Audio processing}
One of the examples, \verb|audioInputExample| demonstrates how users can take in an audio signal and have animations that react to the incoming signal. Of note is the \verb|ofSoundStream| class --- a class that allows the user to access audio devices. In terms of audio processing, FFT is implemented with \verb|ofSoundGetSpectrum|. It is worth noting that, according to the documentation, \verb|ofSoundGetSpectrum| is not implemented on mobile and embedded platforms.\cite{OFsoundstreamFFT} This could be a problem if there are no workarounds. The openFrameworks sound guides did introduce the concept of Constant-Q transforms and, while it does not currently have a proprietary implementation of Constant-Q transforms, it did point to a workaround.\cite{Klapuri} It also highlighted the Music Information Retrieval Evaluation eXchange (MIREX) for updates on audio processing technologies.\cite{mirexWiki}
\subsubsection{Graphics}
openFrameworks allows users to generate both 2D and 3D graphics. Their implementation comes from numerous classes that allow the user to define the shape or points of the graphic that they wish to generate. openFrameworks also provides an API that provides users with simplified functions that access the more common functions of OpenGL. The documentation for the graphics classes seem more comprehensive to that of the sound classes.

\subsection{Creating a new project}
Creating a new project is straightforward. Users use the \verb|projectGenerator| application. It does take awhile for Xcode to build openFrameworks to initialise projects, however. With the provided examples and online documentation, I was able to create a simple animation that reacts to an incoming audio signal.
\begin{figure}
  \begin{subfigure}{0.5\textwidth}
    \includegraphics[width = 0.9\linewidth]{OF/simplesound_soft}
    \caption{Simple, soft sound}
  \end{subfigure} 
 \begin{subfigure}{0.5\textwidth}
    \includegraphics[width = 0.9\linewidth]{OF/simplesound_med}
    \caption{Simple, medium sound}
  \end{subfigure}
  \begin{subfigure}{0.5\textwidth}
    \includegraphics[width = 0.9\linewidth]{OF/complexsound_soft}
    \caption{Complex, soft sound}
  \end{subfigure}
  \begin{subfigure}{0.5\textwidth}
    \includegraphics[width = 0.9\linewidth]{OF/complexsound_perc}
    \caption{Complex, loud sound}
  \end{subfigure}
\end{figure}
The graphics are generated by representing the various amplitudes of samples stored in a buffer. Samples 1 to 256 are represented by circles on an x-axis. Their displacement along the y-axis reflects their corresponding amplitudes. Finally, the overall volume of the buffer is mapped to the radius of the circles. I found this representation surprisingly effective at conveying the timbre and amplitude of the audio signal. Mapping the radius of the circles to the overall volume of the audio input also led to some interesting patterns that conveyed the complexity of an incoming signal by displaying an increasingly chaotic and indiscernable shape.

\subsection{Current implementations}
The following are some interesting projects where openFrameworks was used as a development framework:

\begin{enumerate}
\item fuse*'s \textit{MULTIVERSE}\cite{multiverse}
\item fuse*'s \textit{D\"okk}\cite{dokk}
\item graycode's \textit{\#include red}\cite{includered}
\item Ryoichi Kurokawa's \textit{node 5:5}\cite{KurokawaNode}
\end{enumerate}

\begin{figure}
  \includegraphics[width = 0.9\linewidth]{OF/multiverse1}
  \caption{Screenshot of \textit{MULTIVERSE}\cite{multiverse}}
  \includegraphics[width = 0.9\linewidth]{OF/multiverse2}
  \caption{Screenshot of \textit{MULTIVERSE}\cite{multiverse}}
\end{figure}
I found the visuals in fuse*'s \textit{MULTIVERSE} to be particularly stunning. The visuals in \textit{MULTIVERSE} are generated through implementing particle behaviours in a simulated environment. The particles are programmed to react to each other, as well as to the space around. Different behaviours are exhibited by altering the vector field that the particles are placed in. Shader programs were used to maximise hardware performance and to optimise the graphics pipline on the GPU.\cite{multiverse} Kurokawa's \textit{node 5:5} had a similar aesthetic that was more closely tied to sound. It might be worth taking a look into how to best generate these graphics efficiently.

\subsection{Issues}
From the documentation on their git page states:

\begin{quote}
OF releases are designed to be self-contained. You can put them anywhere on your hard drive, but it's not possible to mix different releases of OF together, so please keep each release (0.8.0, 0.8.1) separate. Projects may generally work from release to release, but this is not guaranteed.\cite{OFgithub}
\end{quote}

This means that, once I start a project, I'll have to stay within that version of openFrameworks. Utilising newer implementations will require careful version control. Finally, there were some issues with compiling openFrameworks on Xcode 10 but a quick look on the Git repository page provided a fix by editing the corresponding config file.\cite{OFissue} I also found out about Sementic Versioning by reading through the project page --- I should considering adhering to it for my project as well.\cite{TomPrestonWerner}

\end{document}
%%% Local Variables:
%%% mode: latex
%%% TeX-master: "../main_framworkreview"
%%% End:
