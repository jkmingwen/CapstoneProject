\documentclass[../initial_thesis.tex]{subfiles}

\begin{document}
\section{Surveys}
% Design of survey
% - Explain quantitative and qualitative sections
% - Link to survey (maybe place in appendix?)
% Results from survey
%% Summarise survey results:
%% - Ratings
%% - Summarise qualitative results

\subsection{Survey design}
The survey was split into 3 sections corresponding to each animation. Before answering the questions from each section, participants were made to watch a video demonstrating the given animation reacting to various recordings of guitar playing. There were four audio tracks, labelled ``Track 1'' - ``Track 4''. The tracks featured different styles of guitar playing in order to demonstrate a range of potential use cases for the animations. A brief description of the tracks follow:

\begin{enumerate}
\item {Fingerstyle guitar with a clean tone. A pulsing bassline interspersed with higher notes to form a melody.} % Think of a more accurate word than "pulsing"
\item {A funk track with a clean tone. It is occasionally accented with brief bursts of a high Em and uses muted strums to add percusiveness to the playing.}
\item {Slow and drawn out chords played with a clean tone and a subtle delay effect. The first beat of each bar begins with a full chord followed by a series of notes from the same chord.}
\item {A rock track played with a distorted tone. The track alternates between two chords, with various runs and accents after each chords.}
\end{enumerate}

The same four tracks were used as audio inputs for all three animations. The resulting animations were recorded and the video of the guitar playing for the given track was overlayed in the bottom right hand corner of the animation. After watching the animation, participants were made to answer a mix of quantitative and qualitative questions on the demonstration they observed. The survey itself can be seen in Appendix \ref{appendix:survey}.

% The questions, along with the question type in parenthesis, follows:
% \begin{enumerate}
% \item {On a scale of 1-10, would [name of animation] be able to capture your attention for a sustained period (the length of an average song) of time? (Scale of 1-10)}
% \item {Please elaborate on which parts of the audio-visualisation you found engaging (Open-ended)}
% \item {How well did the movements and the distortions of the various shapes and forms in [name of animation] capture the characteristics of sounds heard? (Scale of 1-10)}
% \item {Did the visuals react to the audio in the way you expected it to? If not, how did you expect it to react? (Open-ended)}
% \item {Based on your responses to the previous questions, please rank which track is most suited for [name of animation] (\#1 being the most suited) (Ranking)}
% \item {What other aspects of the audio signal do you feel could have been reflected in the visuals? (Open-ended)}
% \end{enumerate}

% The intent of the first two questions was to get a sense of how engaging the given animation was. The intent of the next two questions was to find out how intuitive the audio to visual mapping was in each animation. The fifth question is aimed at discerning if the given animation were particularly suited to a certain type of playing. The last question aimed to get a sense of what the participant felt was missing in terms of visual representation of the audio.

The intent of the survey was to get a sense of how engaging and intuitive the initial audio-visualisation was. The aim was also to get user feedback on which visualisations were particularly suited for a particular style of playing, as represented by the four different tracks, as well as if the user had any suggestions on how to improve the audio-visual mapping. The data from the survey were collected and visualised in order to gain a better understanding of the results.

\subsection{Survey results}

\begin{figure}
  \includegraphics[width = 0.9\linewidth]{Survey/Engagement}
  \caption{Bargraph of ``Engagement'' ratings}
\end{figure}

\begin{figure}
  \includegraphics[width = 0.9\linewidth]{Survey/Intuitivity}
  \caption{Bargraph of ``Intuitivity'' ratings}
\end{figure}

\end{document}
%%% Local Variables:
%%% mode: latex
%%% TeX-master: "../initial_thesis"
%%% End: