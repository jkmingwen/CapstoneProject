\documentclass[../initial_thesis.tex]{subfiles}

\begin{document}
\section{Surveys} \label{sec:surveys}
% Design of survey
% - Explain quantitative and qualitative sections
% - Link to survey (maybe place in appendix?)
% Results from survey
%% Summarise survey results:
%% - Ratings
%% - Summarise qualitative results

In order to assess the response of audiences to the animations, a survey demonstrating the use of each animation was conducted. Survey participants were recruited from Yale-NUS College. Of the 12 participants in the survey, 3 had no prior music experience while the remaining 9 had between 2 to 5 years of music experience. None of the participants had any prior experience designing audio-visuals. The survey was split into 3 sections corresponding to each animation. Before answering the questions from each section, participants were made to watch a video demonstrating the given visual reacting to various recordings of guitar playing. There were four audio tracks, labelled ``Track 1'' to ``Track 4''. Each track featured a different style of guitar playing in order to demonstrate a range of potential use cases for the animations. A brief description of the tracks follow:

\begin{enumerate}
\item {Fingerstyle guitar with a clean tone. A pulsing bassline interspersed with higher notes to form a melody.} % Think of a more accurate word than "pulsing"
\item {A funk track with a clean tone. It is occasionally accented with brief bursts of a high Em and uses muted strums to add percusiveness to the playing.}
\item {Slow and drawn out chords played with a clean tone and a subtle delay effect. The first beat of each bar begins with a full chord followed by a series of notes from the same chord.}
\item {A rock track played with a distorted tone. The track alternates between two chords, with various runs and accents after each chords.}
\end{enumerate}

The same four tracks were used as audio inputs for all three animations and the resulting animations were recorded. In order to simulate its use in a live context, a video taken of the guitar playing for the given track was overlayed at the bottom right hand corner of the animation. After watching the video demonstration, participants were made to answer a mix of quantitative and qualitative questions on the demonstration they observed. The survey itself can be seen in Appendix \ref{appendix:survey}. \par

% The questions, along with the question type in parenthesis, follows:


% Explain Engagement and Intuitiveness
One of the aims of the survey was to get a sense of the levels of ``Engagement'' and ``Intuitiveness'' associated with the three animations. In order to get quantitative feedback on those two concepts, participants were told to provide a rating from 1-10 in response to the following questions:

\begin{enumerate}
\item {On a scale of 1-10, would [name of animation] be able to capture your attention for a sustained period (the length of an average song) of time?}
\item {How well did the movements and the distortions of the various shapes and forms in [name of animation] capture the characteristics of sounds heard?}
\end{enumerate}

To add on to this, participants had to answer open-ended questions to get an idea of which aspects of each animation they felt contributed to a sense of ``Engagement'' and ``Intuitiveness'', as well as to find out if the user had any suggestions on how to improve the audio-visual mappings. The open-ended questions in the survey were as follows:

\begin{enumerate}
\item {Please elaborate on which parts of the audio-visualisation you found engaging}
\item {Did the visuals react to the audio in the way you expected it to? If not, how did you expect it to react?}
\item {What other aspects of the audio signal do you feel could have been reflected in the visuals?}
\end{enumerate}

Finally, the survey was also used to get user feedback on which animations were particularly suited for a particular style of playing. After watching the video demonstration for the particular animation, participants were asked ``Based on your responses to the previous questions, please rank which track is most suited for [name of animation] (\#1 being the most suited)''.

The data from the survey were collected and visualised in order to gain a better understanding of the results. The visualisations of the results are shown in the following section.

\subsection{Survey results}
It was initially expectated that participants would respond better to Circles for Tracks 2 and 4. Out of the three visuals, Circles is the most responsive looking and would thus match the speed and intensity of a funk and rock track. Furthermore, the audio-visual mapping for Circles was the most explicit, and so participants would be able to focus on the relatively more hectic sounding tracks without being overwhelmed by a fungible audio-visual mapping. Clouds and Shimmer were expected to do better for Tracks 1 and 3 for the opposite reason --- the relatively more complex animations would balance out the simple and laid-back tracks. \par

The results of the rankings can be seen in Appendix \ref{appendix:ranking}. Interestingly, the results of the track rankings largely contradict the initial expectations. Most participants ranked Circles as being the most suitable for Track 3 while being the least suitable for Track 4 and Track 1. Similarly, most participants found Clouds to be most suited for Track 3 and less suitable for Track 4. Shimmer was also found to be the most suitable for Track 3, with many finding it less suitable for Track 2.

\subsubsection{Quantitative data}
The ratings for ``Engagement'' and ``Intuitiveness'' are visualised in violin plots as seen in Figure \ref{fig:vplot_ratings}. From Figure \ref{fig:engagement_ratings}, it can be seen that Circles has the highest mean rating, followed by Clouds and Shimmer. The mean ``Engagement'' ratings are 6.75, 5.67, and 4.42 for Circles, Clouds, and Shimmer respectively. We see a similar order of ratings for ``Intuitiveness'' in Figure \ref{fig:intuitive_ratings}. Circles has the highest mean rating of 7.17, followed by Clouds with 5.83, and finally Shimmer with a mean rating of 4.75. Given its clear audio to visual mapping, it was not particularly unusual for Circles to receive a high rating for ``Intuitiveness''. What was surprising, however, was that Circles also received the highest rating for ``Engagement''. These initial results imply that participants found Circles --- which is the least compliant to Sa's visual design principles --- to be the most engaging. In contrast, participants found Shimmer --- which largely adhered to Sa's principles --- to be the least engaging. While adhering to Sa's principles for designing visuals might help to prevent visuals from taking precedence over its accompanying audio, the survey results show that these same visuals might be less capable of capturing the viewer's attention for sustained periods of time. \par

In Appendix \ref{appendix:participant_ratings}, we see how each participant rated the ``Engagement'' and ``Intuitiveness'' of each animation visualised on a grouped bargraph. It is again evident that most participants felt that Circles was the most engaging and intuitive audio-visual mapping. What is particularly interesting is that, with the exception of participants 9 and 11, the ranking of the three animations remain the same in both ``Engagement'' and ``Intuitiveness''. It seems to suggest some kind of correlation between the two concepts.

\begin{figure}
  \begin{subfigure}{0.5\textwidth}
    \centering
    \includegraphics[width = 0.9\linewidth]{Survey/vplot_engagement}
    \caption{``Engagement'' ratings}
    \label{fig:engagement_ratings}
  \end{subfigure}
  \begin{subfigure}{0.5\textwidth}
    \centering
    \includegraphics[width = 0.9\linewidth]{Survey/vplot_intuitiveness}
    \caption{``Intuitiveness'' ratings}
    \label{fig:intuitive_ratings}
  \end{subfigure}
  \caption{Violin plots of participant ratings of animation ``Engagement'' and ``Intuitiveness''}
  \label{fig:vplot_ratings}
\end{figure}

\subsubsection{Qualitative data}
The qualitative data was visualised by tagging the responses to the open-ended questions in the survey with various actionable classifications and visualising the counts of those tags on a radar graph. The definition of these tags can be found in Appendix \ref{appendix:tagdefs}. Figures \ref{fig:engagement_radar}, \ref{fig:intuitiveness_radar}, and \ref{fig:improvement_radar} visualise the results of the comments corresponding to the three open-ended questions. The three open-ended questions could be broadly classified as asking about ``Engagement'', ``Intuitiveness'', and possible ``Improvement''.

\begin{figure}
  \begin{subfigure}{0.5\textwidth}
    \centering
    \includegraphics[width = 0.9\linewidth]{Survey/qual_engagement_circles}
    \caption{Circles}
    \label{fig:engagement_circles}
  \end{subfigure}
  \begin{subfigure}{0.5\textwidth}
    \centering
    \includegraphics[width = 0.9\linewidth]{Survey/qual_engagement_clouds}
    \caption{Clouds}
    \label{fig:engagement_clouds}
  \end{subfigure}
  \begin{subfigure}{0.5\textwidth}
    \centering
    \includegraphics[width = 0.9\linewidth]{Survey/qual_engagement_shimmer}
    \caption{Shimmer}
    \label{fig:engagement_shimmer}
  \end{subfigure}
  \caption{Radar graphs of responses to the question \textit{``Please elaborate on which parts of the audio-visualisation you found engaging''} for each animation}
  \label{fig:engagement_radar}
\end{figure}

\begin{figure}
  \begin{subfigure}{0.5\textwidth}
    \centering
    \includegraphics[width = 0.9\linewidth]{Survey/qual_intuitiveness_circles}
    \caption{Circles}
    \label{fig:intuitiveness_circles}
  \end{subfigure}
  \begin{subfigure}{0.5\textwidth}
    \centering
    \includegraphics[width = 0.9\linewidth]{Survey/qual_intuitiveness_clouds}
    \caption{Clouds}
    \label{fig:intuitiveness_clouds}
  \end{subfigure}
  \begin{subfigure}{0.5\textwidth}
    \centering
    \includegraphics[width = 0.9\linewidth]{Survey/qual_intuitiveness_shimmer}
    \caption{Shimmer}
    \label{fig:intuitiveness_shimmer}
  \end{subfigure}
  \caption{Radar graphs of responses to the question \textit{``Did the visuals react to the audio in the way you expected it to? If not, how did you expect it to react?''} for each animation}
  \label{fig:intuitiveness_radar}
\end{figure}

\begin{figure}
  \begin{subfigure}{0.5\textwidth}
    \centering
    \includegraphics[width = 0.9\linewidth]{Survey/qual_improvement_circles}
    \caption{Circles}
    \label{fig:improvement_circles}
  \end{subfigure}
  \begin{subfigure}{0.5\textwidth}
    \centering
    \includegraphics[width = 0.9\linewidth]{Survey/qual_improvement_clouds}
    \caption{Clouds}
    \label{fig:improvement_clouds}
  \end{subfigure}
  \begin{subfigure}{0.5\textwidth}
    \centering
    \includegraphics[width = 0.9\linewidth]{Survey/qual_improvement_shimmer}
    \caption{Shimmer}
    \label{fig:improvement_shimmer}
  \end{subfigure}
  \caption{Radar graphs of responses to the question \textit{``What other aspects of the audio signal do you feel could have been reflected in the visuals?''} for each animation}
  \label{fig:improvement_radar}
\end{figure}

As expected, as seen in Figure \ref{fig:engagement_radar}, there are a particularly high counts of ``Clarity'' used to describe what made Circles engaging --- this corresponds to the results from the quantitative data. In contrast, it was ``Shape'' and ``Complexity'' that made Clouds and Shimmer engaging. In terms of ``Intuitiveness'', as seen in Figure \ref{fig:intuitiveness_radar}, Shimmer fared the worse, with many counts of ``Unexpected'' and ``Incomplete'' tags --- this also corresponds to the results from the quantitative data. Circles and Clouds, on the other hand, did relatively well with most users finding the visuals reacting in to the audio in the way they expected it to. In terms of possible improvements, we see in Figure \ref{fig:improvement_radar} that in Clouds and Shimmer, participants mostly wanted to see improvements in one or two areas. In contrast, there was a wider variety of areas that participants felt Circles could improve in visualising: one participant suggested that ``gain could be associated with more edged shapes'', while another suggested tracking pitch and using colours to allow more components of the audio signal to be visualised.

The levels of ``Engagement'' of the three animations could be roughly related to the amount of \emph{added value} they felt from the the combination of audio and visuals. One participant mentioned that what they found engaging was ``watching the density of the circles change as they expanded and the extra layer that added to the sound''. On the other hand, the ``Intuitiveness'' of the animation could be seen as a reflection of the \emph{synchresis} felt by the viewer --- their ability to form an association between what they hear and what they see. While Circles seemed to perform the best in those two measures, it was clear from the comments that there were still aspects of the audio that could have been represented in all three animations. Furthermore, from the track rankings, it seemed as though all three animations were suited only for Track 3. The decision was therefore made to keep all three animations and to use the results from the survey to guide the next implementation of audio-visualisations.
\end{document}
%%% Local Variables:
%%% mode: latex
%%% TeX-master: "../initial_thesis"
%%% End: