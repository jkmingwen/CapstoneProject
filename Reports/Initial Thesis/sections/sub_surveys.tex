\documentclass[../initial_thesis.tex]{subfiles}

\begin{document}
\section{Surveys}
% Design of survey
% - Explain quantitative and qualitative sections
% - Link to survey (maybe place in appendix?)
% Results from survey
%% Summarise survey results:
%% - Ratings
%% - Summarise qualitative results

\subsection{Survey design}
The survey was split into 3 sections corresponding to each animation. Before answering the questions from each section, participants were made to watch a video demonstrating the given animation reacting to various recordings of guitar playing. There were four audio tracks, labelled ``Track 1'' to ``Track 4''. Each track featured a different style of guitar playing in order to demonstrate a range of potential use cases for the animations. A brief description of the tracks follow:

\begin{enumerate}
\item {Fingerstyle guitar with a clean tone. A pulsing bassline interspersed with higher notes to form a melody.} % Think of a more accurate word than "pulsing"
\item {A funk track with a clean tone. It is occasionally accented with brief bursts of a high Em and uses muted strums to add percusiveness to the playing.}
\item {Slow and drawn out chords played with a clean tone and a subtle delay effect. The first beat of each bar begins with a full chord followed by a series of notes from the same chord.}
\item {A rock track played with a distorted tone. The track alternates between two chords, with various runs and accents after each chords.}
\end{enumerate}

The same four tracks were used as audio inputs for all three animations. The resulting animations were recorded with the video of the guitar playing for the given track was overlayed at the bottom right hand corner of the animation. After watching the video demonstration, participants were made to answer a mix of quantitative and qualitative questions on the demonstration they observed. The survey itself can be seen in Appendix \ref{appendix:survey}. \par

% The questions, along with the question type in parenthesis, follows:
% \begin{enumerate}
% \item {On a scale of 1-10, would [name of animation] be able to capture your attention for a sustained period (the length of an average song) of time? (Scale of 1-10)}
% \item {Please elaborate on which parts of the audio-visualisation you found engaging (Open-ended)}
% \item {How well did the movements and the distortions of the various shapes and forms in [name of animation] capture the characteristics of sounds heard? (Scale of 1-10)}
% \item {Did the visuals react to the audio in the way you expected it to? If not, how did you expect it to react? (Open-ended)}
% \item {Based on your responses to the previous questions, please rank which track is most suited for [name of animation] (\#1 being the most suited) (Ranking)}
% \item {What other aspects of the audio signal do you feel could have been reflected in the visuals? (Open-ended)}
% \end{enumerate}

% The intent of the first two questions was to get a sense of how engaging the given animation was. The intent of the next two questions was to find out how intuitive the audio to visual mapping was in each animation. The fifth question is aimed at discerning if the given animation were particularly suited to a certain type of playing. The last question aimed to get a sense of what the participant felt was missing in terms of visual representation of the audio.

% Explain Engagement and Intuitiveness
One of the aims of the survey was to get a sense of the levels of ``Engagement'' and ``Intuitiveness'' associated with the three animations. In order to get quantitative feedback on those two concepts, participants were told to provide a rating from 1-10 in response to the following questions:

\begin{enumerate}
\item {On a scale of 1-10, would [name of animation] be able to capture your attention for a sustained period (the length of an average song) of time?}
\item {How well did the movements and the distortions of the various shapes and forms in [name of animation] capture the characteristics of sounds heard?}
\end{enumerate}

To add on to this, participants had to answer open-ended questions to get an idea of which aspects of each animation they felt contributed to a sense of ``Engagement'' and ``Intuitiveness''. The survey was also used to get user feedback on which animations were particularly suited for a particular style of playing, as well as to find out if the user had any suggestions on how to improve the audio-visual mappings. The data from the survey were collected and visualised in order to gain a better understanding of the results. The visualisations of the results are shown in the following section. \par

\subsection{Survey results}
My initial expectation was that participants would respond better to Circles for Tracks 2 and 4. Circles has the more responsive looking animations and would thus match the speed and intensity of a funk and rock track. Furthermore, the audio-visual mapping for Circles was the most explicit, and so participants would be able to focus on the relatively more hectic sounding tracks without being overwhelmed by a fungible audio-visual mapping. I expected Clouds and Shimmer to do better for Tracks 1 and 3 for the opposite reason --- the relatively more complex animations would balance out the more simple and laid-back tracks.

\subsubsection{Quantitative data}
The mean ratings for ``Engagement'' and ``Intuitiveness'' can be seen in Table \ref{tab:meantable}. From the table, it can be seen that Circles has the highest ratings in both categories of ``Engagement'' and ``Intuitiveness'', followed by Clouds and, finally, Shimmer. Given its clear audio to visual mapping, I had expected Circles to receive a high rating for ``Intuitiveness'', but I was surprised by how it also received the highest rating for ``Engagement'' --- I had expected participants to find the visuals of Circles boring in comparison to the more complex audio to visual mappings of Clouds and Shimmer.\par

In Figures \ref{fig:engagement_ratings} and \ref{fig:intuitive_ratings}, we see the ratings given by each participant for the ``Engagement'' and ``Intuitiveness'' of each animation plotted on a grouped bargraph. From these two figures, we can see how each participant rated the ``Engagement'' and ``Intuitiveness'' of each animation. It is again clear that most participants felt that Circles was the most engaging and intuitive audio-visual mapping. What's particularly interesting is that, with the exception of participants 9 and 11, the ranking of the three animations remain the same in both ``Engagement'' and ``Intuitiveness''. It seems to suggest some kind of correlation between the two concepts.

\begin{table}[]
  \centering
\begin{tabular}{|l|l|l|l|}
\hline
\rowcolor[HTML]{FFCE93} 
            & Circles & Clouds & Shimmer \\ \hline
Engagement  & 6.75    & 5.67   & 4.42    \\ \hline
Intuitivity & 7.17    & 5.83   & 4.75    \\ \hline
\end{tabular}
\caption{Mean ratings of ``Engagement'' and ``Intuitiveness'' of initial audio-visualisations}
\label{tab:meantable}
\end{table}

\begin{figure}
  \includegraphics[width = 0.9\linewidth]{Survey/Engagement}
  \caption{Bargraph of ``Engagement'' ratings}
  \label{fig:engagement_ratings}
\end{figure}

\begin{figure}
  \includegraphics[width = 0.9\linewidth]{Survey/Intuitivity}
  \caption{Bargraph of ``Intuitiveness'' ratings}\
  \label{fig:intuitive_ratings}
\end{figure}

\subsubsection{Qualitative data}
The qualitative data was visualised by tagging the comments with various actionable classifications and visualising the counts of those tags on a radar graph. The definition of these tags can be found in Appendix. %add appendix!
Figures \ref{fig:engagement_radar}, \ref{fig:intuitive_radar}, and \ref{fig:improvement_radar} respectively visualise the results of the comments corresponding to the following three open-ended questions:

\begin{enumerate}
\item {Please elaborate on which parts of the audio-visualisation you found engaging}
\item {Did the visuals react to the audio in the way you expected it to? If not, how did you expect it to react?}
\item {What other aspects of the audio signal do you feel could have been reflected in the visuals?}
\end{enumerate}

\begin{figure}
  \includegraphics[width = 0.9\linewidth]{Survey/qual_Engagement}
  \caption{Radar graph of comments on ``Engagement''}
  \label{fig:engagement_radar}
\end{figure}

\begin{figure}
  \includegraphics[width = 0.9\linewidth]{Survey/qual_Intuitive}
  \caption{Radar graph of comments on ``Intuitiveness''}
  \label{fig:intuitive_radar}
\end{figure}

\begin{figure}
  \includegraphics[width = 0.9\linewidth]{Survey/qual_improve}
  \caption{Radar graph of comments on ``Improvements''}
  \label{fig:improvement_radar}
\end{figure}

As expected, as seen in Figure \ref{fig:engagement_radar}, there are a particularly high counts of ``Clarity'' used to describe what made Circles engaging. In contrast, it was ``Shape'' and ``Complexity'' that made Clouds and Shimmer engaging. In terms of ``Intuitiveness'', as seen in Figure \ref{fig:intuitive_radar}, Shimmer fared the worse, with many counts of ``Unexpected'' and ``Incomplete'' classifications while Circles and Clouds did relatively well with most users finding the visuals reacting in to the audio in the way they expected it to. In terms of possible improvements, we see in Figure \ref{fig:improvement_radar} that in Clouds and Shimmer, participants mostly wanted to see improvements in one or two areas, while there was a wider variety of areas that participants felt Circles could improve in visualising.\par

These results will be used guide the next implementation of audio-visualisations. Following which, the various audio-visualisations will be implemented on an SBC, as reviewed in Section \ref{sec:boardreview}, before being benchmarked. 
\end{document}
%%% Local Variables:
%%% mode: latex
%%% TeX-master: "../initial_thesis"
%%% End: