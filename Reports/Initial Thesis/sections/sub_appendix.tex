\documentclass[../initial_thesis.tex]{subfiles}

\begin{document}
\chapter{Survey Form} \label{appendix:survey}
\includepdf[pages=-]{Survey/CapstoneSurveyForm.pdf}

\chapter{Additional Survey Results}
\section{Track Ranking} \label{appendix:ranking}
\begin{figure}[!htb]
  \centering
  \includegraphics[width = \textwidth]{Survey/circles_rank}
\end{figure}
\begin{figure}[!htb]
  \centering
  \includegraphics[width = \textwidth]{Survey/clouds_rank}
\end{figure}
\begin{figure}[!htb]
  \centering
  \includegraphics[width = \textwidth]{Survey/shimmer_rank}
\end{figure}

\section{``Engagement'' and ``Intuitiveness'' ratings} \label{appendix:participant_ratings}
\begin{figure}[!htb]
  \centering
  \includegraphics[width = \textwidth]{Survey/Engagement}
\end{figure}
\begin{figure}[!htb]
  \centering
  \includegraphics[width = \textwidth]{Survey/Intuitivity}
\end{figure}

\section{Comment tag definitions} \label{appendix:tagdefs}
\paragraph{Engagement}
\begin{itemize}
\item {\emph{Shape}: The participant found changes in the various geometric aspects of the shapes in the animation engaging}
\item {\emph{Colour}: The participant found changes in the RGBA values of the shapes in the animation engaging}
\item {\emph{Clarity}: The participant found the audio-visualisation engaging when it was easily understandable, with clear changes in terms of how the visuals evolved over time}
\item {\emph{Incoherent}: The participant found that they could not understand what was going on in the audio-visualisation – this usually meant that there was too much stimuli in some aspect of the audio-visualisation}
\item {\emph{Complexity}: The participant found the complexity of the animation, in the form of fungible audio-visual mapping or independent movements, engaging}
\item {\emph{Responsive}: The participant found the responsiveness of the visuals to the audio engaging}
\end{itemize}

\paragraph{Intuitiveness}
\begin{itemize}
\item {\emph{Expected}: The participant found the way the animation reacted to the audio signal intuitive}
\item {\emph{Unexpected}: The participant found the way the animation reacted to the audio signal unintuitive}
\item {\emph{Unclear mapping}: The participant could not discern the audio-visual mapping}
\item {\emph{Incomplete}: The participant felt that the audio-visual mapping did not fully capture the characteristics of the audio signal}
\item {\emph{Incoherent}: The participant was not able to comprehend the movements of the animation – this usually means that the animation was too messy to understand}
\end{itemize}

\paragraph{Improvements}
\begin{itemize}
\item {\emph{Pitch}: Visuals could be improved by reflecting changes in notes played – be it a single note or a chord}
\item {\emph{Colours}: Participant thought that audio-visualisation could be improved by adding colours}
\item {\emph{Timbre}: Participant felt that audio-visualisation could be improved by visualising timbre of the audio}
\item {\emph{Responsiveness}: Participant felt that visuals could be improved by making the changes in the visuals more sensitive to changes in the audio signal – usually means that participant felt that the visuals did not seem particularly reactive to a the audio}
\item {\emph{Genre}: Visuals could be improved by including a means of differentiating between playing styles – be it genre or playing technique}
\item {\emph{Clarity}: Visuals could be improved by making it more clear – this usually means that the participant felt that there were too many moving elements in the visual to make out how it was being affected by the audio}
\item {\emph{Tempo}: Visuals could be improved by reflecting the tempo of the track}
\end{itemize}
\end{document}
%%% Local Variables:
%%% mode: latex
%%% TeX-master: "../../initial_thesis"
%%% End:
