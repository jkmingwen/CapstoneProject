\documentclass[../../initial_thesis.tex]{subfiles}

\subsection{Teenage Engineering \textit{OP-Z}}
\begin{document}
Teenage Engineering's \textit{OP-Z} is a sequencer and multimedia synthesizer with the ability to produce accompanying visuals \cite{Teenageengineering}. In contrast to the \textit{ETC} and the \textit{SYNCHRONATOR HD}, the \textit{OP-Z} has an additional primary function of producing audio output. Notably, video ouput is not generated by the \textit{OP-Z} itself, but through an application running on an iOS device \cite{Sonicstate2018}.\footnotemark The \textit{OP-Z} application has two features that allow users to link their audio to visuals --- Photomatic and Motion. Photomatic allows users to sequence still images and to apply effects onto those images. Motion, on the other hand, allows users to display and control live 2D or 3D visuals --- notably, users have the ability to create their own visuals using the Unity engine \cite{Teenageengineering}. It is therefore able to produce 3D graphics --- something that neither the \textit{SYNCHRONATOR HD} nor the \textit{ETC} are capable of. Figure \ref{fig:opz_interface} shows the interface of the \textit{OP-Z}. \par
\footnotetext{At the synthesizer show, Superbooth 2018, a representative of the company states (at 2:20) that ``The OP-Z itself isn't putting out any visuals, it's just talking to the app and controlling the app''.}

\begin{figure}
  \begin{subfigure}{0.5\textwidth}
    \includegraphics[width = 0.9\linewidth]{Report2/opz_motion}
    \caption{Visuals from Motion \cite{Sonicstate2018}}
    \label{fig:opz_visuals}
  \end{subfigure}
  \begin{subfigure}{0.5\textwidth}
    \includegraphics[width = 0.9\linewidth]{Report2/opz}
    \caption{The \textit{OP-Z} interface \cite{Teenageengineering}}
    \label{fig:opz_interface}
  \end{subfigure}
  \caption{\textit{OP-Z} visuals and interface}
  \label{fig:opz}
\end{figure}

An important distinction to make is that visuals generated by the \textit{OP-Z} are not directly reacting to the audio signal produced by the \textit{OP-Z}. With Photomatic, the sequencing of each image is synchronised to a step sequencer that the \textit{OP-Z} also uses to time the beats of its audio output. The visuals on Motion, while utilising the Unity engine, need to be produced using \textit{videolab} --- a toolkit developed by Teenage Engineering that allows users to create MIDI-controlled video content \cite{videolabgit}. An example of which can be seen in Figure \ref{fig:opz_visuals}. This means that Motion displays visuals that are triggered by MIDI messages rather than reacting to information extracted from the audio signal. The \textit{OP-Z} itself acts as a device that triggers changes in the video output by sending the iOS device MIDI messages. This means that, unlike the \textit{ETC} and the \textit{SYNCHRONATOR HD}, the \textit{OP-Z} does not utilise any low-level audio features directly from an audio signal. \par

While it does not directly deal with live audio signals, Teenage Engineering's \textit{OP-Z} demonstrates how it is possible to use extract music information from music notation --- in the form of MIDI messages --- to synchronise audio to visuals. In fact, the generation of the \textit{OP-Z}'s visuals might be less computationally intensive as it does not involve processing an audio signal to retrieve music information. That said, because it relies on information conveyed through MIDI messages, such a system is unable to have its visuals react to live audio input from an external instrument that does not output MIDI information --- this excludes the majority of instruments that do not utilise a MIDI controller from interfacing with the \textit{OP-Z}'s system of producing video output.

% From Superbooth 2018 video: https://www.youtube.com/watch?v=MVBRIkuzsG8 at about 2:20
% OP-Z visuals: (1) Photomatic - sequencing of photos in synch to music
% The OP-Z itself doesn't output any video, it simply sends messages to the app that then generates those videos
% Teenage Engineering intro to Motion: https://www.youtube.com/watch?v=LPpFP2QUMX4
% OP-Z visuals: (2) Motion - 2D or 3D graphics controlled by the OP-Z; key word here is "controlled" - it relies on the sequencer to trigger parts of the video


\end{document}
%%% Local Variables:
%%% mode: latex
%%% TeX-master: "../../initial_thesis"
%%% End: