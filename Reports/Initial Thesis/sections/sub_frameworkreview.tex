\documentclass[../initial_thesis.tex]{subfiles}

\begin{document}
\section{Framework review}\label{sec:frameworkreview}
The purpose of the framework review is to assess the viability of the various frameworks available to implement this project on. As the goal is to build a video synthesizer for audiovisual expression, the frameworks need to facilitate programming animations. They will also need to facilitate taking in an audio signal as a parameter to affect the animations in real time. With that said, it would be impractical to attempt to assess every single framework available. As stated above, the frameworks considered were therefore selected based on their intended use cases in relation to this project: processing audio signals and producing visual output. openFrameworks and Cinder were chosen due to their prominence as frameworks used in implementing interactive visuals. JUCE was chosen due to its focus on developing music applications. OPENRNDR was chosen as it is a promising new framework focused on visual computing. Finally, FLAM3 was chosen due to the unique and complex visuals it was capable of generating, as seen in Figure \ref{fig:flames}. Thus, this framework review will be focused on these five frameworks as listed in Table \ref{tab:frameworktable}. 

\begin{table}
\footnotesize
\centering
\resizebox{\textwidth}{!}{
\begin{tabular}{llcl}
\hline
\rowcolor[HTML]{FFCE93} 
Framework      & Graphics & \multicolumn{1}{l}{\cellcolor[HTML]{FFCE93}\begin{tabular}[c]{@{}c@{}}Native Audio\\ API\end{tabular}} & Platform                                                                                       \\ \hline
openFrameworks & 2D, 3D   & Y                                                                                                      & \multicolumn{1}{c}{\begin{tabular}[c]{@{}l@{}}MacOS/Windows/\\ Linux/iOS/Android\end{tabular}} \\ \hline
Cinder         & 2D, 3D   & Y                                                                                                      & \begin{tabular}[c]{@{}l@{}}MacOS/Windows/\\ Linux\end{tabular}                                 \\ \hline
JUCE           & 2D, 3D   & Y                                                                                                      & \begin{tabular}[c]{@{}l@{}}MacOS/Windows/\\ Linux\end{tabular}                                 \\ \hline
OPENRNDR       & 2D       & N                                                                                                      & \begin{tabular}[c]{@{}l@{}}MacOS/Windows/\\ Linux\end{tabular}                                 \\ \hline
FLAM3          & 2D       & N                                                                                                      & Windows/Linux                                                                                  \\ \hline
\end{tabular}%
}
\caption{An overview of frameworks tested}
\label{tab:frameworktable}
\end{table}

The frameworks will be assessed upon the following factors: the \emph{learning curve} of the framework; its \emph{popularity}; as well as the related \emph{existing projects} using them. It is important that the learning curve of the given framework is not too steep as efforts on learning a framework cannot take priority over the progress of the project as a whole. The popularity of the framework could act as an indicator of how easy it is to pick up and, more importantly, the chance of having compatible code with future collaborators. Studying its implementations in existing projects would demonstrate the capabilities of each framework. It would also give us an idea of the types of implementations the each framework is geared towards. Researching exisiting audiovisual projects that utilise each framework could also provide inspiration about how one could go about implementing a video synthesizer. The goal of this review is to come to a decision on which framework to use for the duration of this project --- these assessments would help to make the best choice.

\subfile{subsections/sub_of}
\subfile{subsections/sub_cinder}
\subfile{subsections/sub_juce}
\subfile{subsections/sub_openrndr}
\subfile{subsections/sub_flam3}

\subsection{Deciding on a framework}
After working with these frameworks, openFrameworks comes across as the most viable for several reasons. The first reason is its popularity. openFrameworks is the most popular of the frameworks chosen for review --- consequentially, its forums are much more active and thus offer a larger community to look towards for both inspiration and help. This is what sets it apart from Cinder, which is the framework most comparable to openFrameworks. While JUCE would likely be a more suitable candidate if my focus were purely on audio processing, the framerate issues with the simple visualiser implementation in JUCE makes the framework less suitable for the more complex animations planned for this project. In contrast, openFrameworks provides the graphics APIs necessary to produce complex animations that react to an audio input. Finally, while promising, OPENRNDR is still a new framework without an API to handle audio processing. FLAM3 would not be a suitable framework for the reasons described in its corresponding ``Issues'' section. Therefore, with the experiences gained working with these five frameworks, openFrameworks was chosen as the framework used to develop this project.
\end{document}
%%% Local Variables:
%%% mode: latex
%%% TeX-master: "../initial_thesis"
%%% End:
