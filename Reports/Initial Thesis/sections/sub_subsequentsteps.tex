\documentclass[../initial_thesis.tex]{subfiles}

\begin{document}
\section{Subsequent steps}
% Write a premilinary work section
% This is what I've done:
% Audio parameter (amplitude) in time-domain to visual parameters
% Real-time Hi-Z input into computer

% Note: there's a lot of personal pronouns in this section --- consider changing them to fit the tone of the paper

At this point, let us briefly recap the preliminary research that has been done and how this will be used in the execution of this project. Thus far, we have succeeded in mapping an audio parameter in the time-domain (amplitude) to visual parameters on a computer. The results of which can be seen in sections \ref{sec:openframeworks}, \ref{sec:cinder}, and \ref{sec:juce}. This mapping is performed with a real-time audio signal as input.\footnotemark These implementations can all take in the audio signal from an electric guitar. \par
\footnotetext{A demonstration of the implementation on openFrameworks can be viewed here: \url{https://vimeo.com/301165633}.}

The information gained from the review of the field of audiovisual expression, music feature extraction, and video synthesizers, as well as the framework and board reviews, are the starting points for the next steps of this project. The first task would be to design a visual language --- a mapping of forms of visual movement to various low-level audio features, such as those listed in Section \ref{sec:lowlevelaudio} --- for the video synthesizer. The design of this visual language will be guided by the works of John Whitney, Michel Chion, Adriana Sa, as well as some of the current implementations highlighted in the framework review. Once this has been decided upon, work will begin in implementing this visual language using the chosen framework --- openFrameworks --- from Section \ref{sec:frameworkreview}, in a way that will respond to the parameters of a live audio input. The next logical steps after this is to embark upon user testing to shape iterative prototypes of the visual language, ultimately followed by implementing the video synthesizer on a SBC. The selection of the SBC to use will be guided by the board review in Section \ref{sec:boardreview}. The interface of the final product will also undergo a series of iterative prototypes guided by feedback from user tests.

\end{document}
%%% Local Variables:
%%% mode: latex
%%% TeX-master: "../initial_thesis"
%%% End: