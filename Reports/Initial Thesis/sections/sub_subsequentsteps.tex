\documentclass[../initial_thesis.tex]{subfiles}

\begin{document}
% Write a premilinary work section
% This is what I've done:
% Audio parameter (amplitude) in time-domain to visual parameters
% Real-time Hi-Z input into computer

Thus far, I have succeeded in mapping an audio parameter in the time-domain (amplitude) to visual parameters on a computer. The results of which can be seen in sections \ref{sec:openframeworks}, \ref{sec:cinder}, and \ref{sec:juce}. This mapping is performed with a real-time audio signal as input.\footnotemark These implementations can all take in the audio signal from an electric guitar. \par
\footnotetext{A demonstration of the implementation on openFrameworks can be viewed here: \url{https://vimeo.com/301165633}.}


The information gained from the review of the field of audiovisual expression, music feature extraction, and video synthesizers, as well as the framework and board reviews, are the starting points for the next steps of this project. The first task following this report would be to design a visual language --- in the form of a mapping of forms of visual movement to characteristics of audio --- for the video synthesizer. The design of this visual language will be guided by the works of John Whitney, Michel Chion, Adriana Sa, as well as some of the current implementations highlighted in the framework review. Once this has been decided upon, work will begin in implementing this visual language using the chosen framework from Section \ref{sec:frameworkreview}, in a way that will respond to the parameters of a live audio input. The next logical steps after this is to embark upon user testing to shape iterative prototypes of the visual language, ultimately followed by implementing the video synthesizer on a SBC. The selection of the SBC to use will be guided by the board review in Section \ref{sec:boardreview}. The interface of the final product which will also undergo a series of iterative prototypes guided by feedback from user tests.
% What is the objective of this project? If it's an art work, then user testing only comes after implemntation in hardware <-- maybe this more?. If it's supposed to be a psychological thing, then I test right after implementing audiovisual language.
\end{document}
%%% Local Variables:
%%% mode: latex
%%% TeX-master: "../initial_thesis"
%%% End: