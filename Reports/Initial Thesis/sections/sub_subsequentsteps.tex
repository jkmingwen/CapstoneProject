\documentclass[../initial_thesis.tex]{subfiles}

\begin{document}
\section{Introduction}
% Write a premilinary work section
% This is what I've done:
% Audio parameter (amplitude) in time-domain to visual parameters
% Real-time Hi-Z input into computer

% Note: there's a lot of personal pronouns in this section --- consider changing them to fit the tone of the paper

The information gained from the review of the field of audiovisual expression, music feature extraction, and video synthesizers, as well as the framework and board reviews, are the starting points for the next steps of this project. The first task would be to design a visual language: a mapping of forms of visual movement to various low-level audio features, such as those listed in Section \ref{sec:lowlevelaudio}, for the video synthesizer. The design of this visual language will be guided by the works of John Whitney, Michel Chion, Adriana Sa, as well as some of the current implementations highlighted in the framework review. Once this has been decided upon, work will begin in implementing this visual language using openFrameworks: the chosen framework from Section \ref{sec:frameworkreview}. The visuals will respond to the parameters of a live audio input, therefore forming an audiovisualisation. The next step is to embark upon user testing to shape iterative prototypes of the visual language, ultimately followed by implementing the video synthesizer on the Odroid XU4 as decided upon in Section \ref{sec:boardreview}. Having done so, the latency of audio input to visual output of the video synthesizer on the SBC will be measured.

\end{document}
%%% Local Variables:
%%% mode: latex
%%% TeX-master: "../initial_thesis"
%%% End: