\documentclass[../initial_thesis.tex]{subfiles}

\begin{document}
\section{Implementing animations}
There were three animations implemented. For ease of reference, they have been named according to their characteristic shape or movements: Circles, Clouds, and Shimmer.

% Include screenshots of animations
\subsection{Circles}
Circles is made up of a series of ellipses placed adjacent to one another to form a horizontal line in the middle of the screen. It is based off the visualisation of an audio signal that was first implemented in Section \ref{sec:openframeworks}. The mapping of amplitude to displacement along the y-axis, as well as volume to radius of ellipse, remains the same.

\subsection{Clouds}
Clouds is based on Konstantine Wegner's algorithm in ``Wobbly Swarm''. His algorithm forms the basis for a basic particle cloud in which each particle, represented by an ellipse, experiences an attractive force to every other particle until they cross a threshold of being too close too one another --- after which, they experience a repulsive force pushing them away from one another.

\subsection{Shimmer}
Shimmer utilises lines that spin about a fixed axis to form a ripple effect that slowly spreads out across the screen. It is distinct from the other two animations as there isn't one fixed point of interest for the viewer to focus on --- all elements in the screen are constantly in motion. Perlin Noise is used to determine the rotation of these lines.
\end{document}
%%% Local Variables:
%%% mode: latex
%%% TeX-master: "../initial_thesis"
%%% End: