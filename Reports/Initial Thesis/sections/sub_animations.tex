\documentclass[../initial_thesis.tex]{subfiles}

\begin{document}
\section{Initial animations}
There were three animations implemented. For ease of reference, they have been named according to their characteristic shape or movements: Circles, Clouds, and Shimmer.

% Include screenshots of animations
\subsection{Circles}
Circles is made up of a series of ellipses placed adjacent to one another to form a horizontal line in the middle of the screen. Each ellipse corresponds to a sample in the buffer. It is based off the visualisation of an audio signal that was first implemented in Section \ref{sec:openframeworks}. The mapping of amplitude to the ellipse's displacement along the y-axis, as well as volume to radius of the ellipses, remains the same. The design of this animation was inspired by Chion's writing on temporalising an image with sound, where he states that ``A smooth and continuous sound is less animating than an uneven or fluttering one'' \cite{Chion1994}. As seen in Figure \ref{fig:circles1}, by mapping the amplitude of each sample to the displacement of an ellipse along the y-axis, Circles is able to visualise a smooth sound with a wave with regular peaks and troughs and a more complex sound with a correspondingly complex wave shape. The mapping of volume to the radii of the ellipses emphasises the regularity or irregularity of the waveform.

\begin{figure}
  \begin{subfigure}{0.5\textwidth}
    \includegraphics[width = 0.9\linewidth]{Survey/circles_complexsoft}
    \caption{Complex but soft sound}
  \end{subfigure} 
  \begin{subfigure}{0.5\textwidth}
    \includegraphics[width = 0.9\linewidth]{Survey/circles_simpleloud}
    \caption{Simple but loud sound}
  \end{subfigure}
  \caption{Circles reacting to different sounds}
  \label{fig:circles1}
\end{figure}

\subsection{Clouds}
Clouds is based on Konstantin Makhmutov's algorithm in ``Wobbly Swarm'' \cite{Makhmutov}. His algorithm forms the basis for a basic particle cloud in which each particle, represented by an ellipse, experiences an attractive force to every other particle on the screen until they cross a certain threshold of distance --- after which, they experience a repulsive force pushing them away from one another. Here, the volume of the audio signal has been mapped to amplify the force enacted on the particle by the other particles. The amplification of force is also visualised by a temporary increase in the radius of the ellipse. The effect of this mapping can be seen in Figure \ref{fig:clouds1}. In contrast to Circles, Clouds features elements whose movements are not entirely dependent on the audio signal. The decision to include movements in the animation that were independent of the audio signal was primarily informed by Sa's research which concluded that ``perception continues to acknowledge conflicting information, embracing convergences and divergences as inconclusive concepts'' in a fungible audio-visual mapping \cite{Sa2014}. The elements aren't fixed to a specific point on the x-axis or y-axis either. They therefore form a slightly more dynamic system in constrast to Circles.

\begin{figure}
  \begin{subfigure}{0.5\textwidth}
    \includegraphics[width = 0.9\linewidth]{Survey/clouds_soft}
    \caption{Soft sound}
  \end{subfigure} 
  \begin{subfigure}{0.5\textwidth}
    \includegraphics[width = 0.9\linewidth]{Survey/clouds_loud}
    \caption{Loud sound}
  \end{subfigure}
  \caption{Clouds reacting to different sounds}
  \label{fig:clouds1}
\end{figure}

\subsection{Shimmer}
Shimmer utilises lines that spin about an array of fixed points to form a ripple effect, as seen in Figure \ref{fig:shimmer1}, that slowly spreads out across the screen. The angle of each line's rotation is determined by Perlin Noise. Perlin Noise is also used to determine the alpha level, which affects the opacity, of each line. The volume of the audio signal determines the speed of the rotation. It is distinct from the other two animations as there isn't one fixed point of interest for the viewer to focus on --- all elements are spread across the screen and constantly in motion. This animation was similarly informed by Sa's research on how clear cause-effect relationships can lead to a disproportionate focus on the visual element over the aural element of an audio-visualisation \cite{Sa2014}.

\begin{figure}
  \begin{subfigure}{0.5\textwidth}
    \includegraphics[width = 0.9\linewidth]{Survey/shimmer1}
  \end{subfigure} 
  \begin{subfigure}{0.5\textwidth}
    \includegraphics[width = 0.9\linewidth]{Survey/shimmer2}
  \end{subfigure}
  \caption{Shimmer reacting to different sounds}
  \label{fig:shimmer1}
\end{figure}
\end{document}
%%% Local Variables:
%%% mode: latex
%%% TeX-master: "../initial_thesis"
%%% End: