\documentclass[../initial_thesis.tex]{subfiles}

\begin{document}
\section{Board review}\label{sec:boardreview}
One of the goals of this project is to implement a video synthesizer on a portable interface: this will require the use of a Single Board Computer (SBC). SBCs are complete computers --- built on a single circuit board --- distinguished by having built in general purpose I/O ports (GPIO), low power consumption, and low cost \cite{Johnston2018}. The low cost and power consumption of SBCs make them suited for a project such as this where both standard computers and microcontrollers are unviable due to size and cost constraints coupled with moderately demanding processing requirements \cite{Johnston2018}. The purpose of this section is to provide an overview of the technical specifications that were considered when considering the various SBCs available in the market. \par
% Furthermore, understanding the processing and graphics capabilities of the various boards in the market will prevent any unexpected pitfalls when implementing the finalised video synthesizer code on the SBC --- that is, by recognising the limits of the boards available to us, we can calibrate our expectations of the kinds of visuals that this project's video synthesizer can expect to produce. \par

The boards chosen for review are a small sample of those currently available on the market. An overview of their capabilities is available in Table \ref{tab:boardtable}. While they are built to fulfill similar purposes, each board uses a system on a chip (SoC) manufactured by a different company. As a result, they have varying degrees of processing power in terms of CPU and GPU. The goal with this review is to identify the kinds of boards that would be viable for this project. The factors that will determine the viability of a board are its processing capabilities, its graphics processing capabilities, the presence of HDMI and USB ports, and its dimensions. \par

One way of determining a board's processing and graphics processing capabilities are the number of processing cores on its CPU as well as the clock speed of the CPU and the GPU. The amount of random-access memory (RAM) available on the board is also important as it will determine the amount of processing that the SBC can perform before having to revert to swapping out processes from main memory --- a process that tends to incur overheads and slows down the running of the program. The processing capabilities of the SBC is important to consider as digital audio processing tends to be computationally intensive --- requiring a sample rate of about 44100Hz. This, coupled with the generation of complex graphics, means that a CPU and GPU that is capable of performing these operations without introducing too much latency needs to be selected. It is also important that the board has USB and HDMI ports as these are the physical connections with which it will take in an audio input and output generated visual signals respectively. Finally, the dimensions of each board was noted as the size of the board will affect the portability of the final product. \par

The most objective metric of processing and graphic processing capabilities is floating point operations per second. Unfortunately, the only reliable way measure this is to take the measurements ourselves. Given the time and budget constraints of the project, this is not currently viable. Once a program is running, however, we could consider properly benchmarking the program on the various SBCs if they are available to us.

\begin{table}
\resizebox{\textwidth}{!}{%
\begin{tabular}{llllcl}
\hline
\rowcolor[HTML]{FFCE93} 
Board                                                             & CPU                                                                        & GPU                                                                           & RAM   & HDMI & USB      \\ \hline
\begin{tabular}[c]{@{}l@{}}Raspberry Pi 3\\ Model B+\end{tabular} & 4-core @ 1.4GHz                                                            & \multicolumn{1}{c}{N.A.}                                                      & 1GB   & Y    & 2.0      \\ \hline
Jetson TX2                                                        & \begin{tabular}[c]{@{}l@{}}4-core @ 2GHz\\ 2-core @ 2GHz\end{tabular}      & \begin{tabular}[c]{@{}l@{}}NVIDIA Pascal\\ 256-core \\ @ 1300MHz\end{tabular} & 8GB   & Y    & 2.0/3.0  \\ \hline
HiKey 970                                                         & \begin{tabular}[c]{@{}l@{}}4-core @ 2.36GHz\\ 4-core @ 1.3GHz\end{tabular} & \begin{tabular}[c]{@{}l@{}}ARM Mali \\ G72 MP12 \\ @ 850MHz\end{tabular}      & 6GB   & Y    & 2.0/3.0  \\ \hline
Odroid XU4                                                        & \begin{tabular}[c]{@{}l@{}}4-core @ 2GHz\\ 4-core @ 1.3GHz\end{tabular}    & \begin{tabular}[c]{@{}l@{}}ARM Mali\\ T628 MP6\\ @ 600MHz\end{tabular}        & 2GB   & Y    & 2.0/3.0  \\ \hline
Nucleo-144                                                        & 1-core @ 216MHz                                                            & \multicolumn{1}{c}{N.A.}                                                      & 256kB & N.A. & Micro-AB \\ \hline
\begin{tabular}[c]{@{}l@{}}ZCU102\\ Evaluation Kit\end{tabular}   & \begin{tabular}[c]{@{}l@{}}4-core @ 1.5GHz\\ 2-core @ 600MHz\end{tabular}  & \begin{tabular}[c]{@{}l@{}}ARM Mali-400\\ @ 667MHz\end{tabular}               & 512MB & Y    & 2.0/3.0  \\ \hline
\end{tabular}%
}
\caption{Overview of board specifications}
\label{tab:boardtable}
\end{table}

\subfile{subsections/sub_raspberrypi}
\subfile{subsections/sub_jetsontx2}
\subfile{subsections/sub_hikey970}
\subfile{subsections/sub_odroidxu4}
\subfile{subsections/sub_nucleo144}
\subfile{subsections/sub_zcu102}

\subsection{Deciding on a SBC}
From the considerations above, the choice of SBC for this project was narrowed down to the Jetson TX2, the HiKey 970, and the Odroid XU4. Out of these three SBCs, the Odroid XU4 has the smallest dimensions. More importantly, while the Jetson TX2 and the HiKey 970 both have higher processing powers compared to the Odroid XU4, they also cost considerably more than the Odroid XU4. The Jetson TX2 and the HiKey 970 cost \$599 and \$299 respectively --- the Odroid XU4, on the other hand, currently costs \$59. Therefore, having found its processing capabilities sufficient for the purpose of this project, the Odroid XU4 was chosen as the SBC to implement the video synthesizer on due to its relative portability and affordability in contrast to other SBCs with similar specifications.
\end{document}
%%% Local Variables:
%%% mode: latex
%%% TeX-master: "../initial_thesis"
%%% End:
