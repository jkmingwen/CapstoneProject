\documentclass[../../main_report1.tex]{subfiles}

\subsection{Teenage Engineering \textit{OP-Z}}
\begin{document}
Teenage Engineering's \textit{OP-Z} is an audio synthesizer with the ability to produce accompanying visuals. The \textit{OP-Z} allows users to generate and create their own visualisations using the Unity engine \cite{Teenageengineering}. It is therefore able to produce 3D graphics --- something that neither the \textit{SYNCHRONATOR HD} nor the \textit{ETC} are capable of. Figure \ref{fig:opz} shows the interface of the \textit{OP-Z}. \par
\begin{figure}
  \includegraphics[width = 0.9\linewidth]{Report1/opz}
  \caption{The \textit{OP-Z} interface \cite{Teenageengineering}}
  \label{fig:opz}
\end{figure}
The \textit{OP-Z}, in contrast to the \textit{ETC} and the \textit{SYNCHRONATOR HD}, is meant to be an all-in-one synthesizer with the ability to sample tracks, synthesize audio, and to generate visuals to accompany audio output.  The \textit{OP-Z} has a particularly wide array of functions and it differs from the previous two products in that its primary function is producing audio output rather than visual output. That said, the video synthesizer is integrated into the \textit{OP-Z} and this means that it could be seen as less flexible than the \textit{ETC} and the \textit{SYNCHRONATOR HD} as it responds only to the audio of the \textit{OP-Z}. The \textit{OP-Z} has yet to be commercially released, however, and thus the extent to which it is able to interact with external audio sources remains to be seen.
\end{document}