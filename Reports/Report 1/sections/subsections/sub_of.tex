\documentclass[../frameworkreview.tex]{subfiles}

\begin{document}

\subsection{openFrameworks}

\subsubsection{Learning and exploring capabilities}
openFrameworks has an online guide called ofBook to provide a starting point for users new to their various proprietary classes. That said, a lot of the examples in the chapter on sound uses deprecated code \cite{Carlucci}. This meant that it was necessary to cross reference the examples with those provided on the Git repository to fully understand what was going on. That said, ofBook is an ongoing project with its writers inviting readers to post issues on the Git repository page; these guides would therefore likely improve with time. openFrameworks also has a comprehensive documentation page of their proprietary classes and functions \cite{openFrameworks}. These proprietary classes can be identified by the \verb|of| prefix --- for example, \verb|ofBaseApp| provides us with functions that deal with user input.

\paragraph{Audio processing}
One of the examples, \verb|audioInputExample| demonstrates how users can take in an audio signal and have animations that react to the incoming signal. Of note is the \verb|ofSoundStream| class --- a class that allows the user to access audio devices. In terms of audio processing, fast Fourier Transforms (FFT) are implemented with \verb|ofSoundGetSpectrum|. It is worth noting that, according to the documentation, \verb|ofSoundGetSpectrum| is not implemented on mobile and embedded platforms \cite{OFsoundstreamFFT}. This could be a problem if there are no workarounds. A demonstration of the comprehensiveness of the openFrameworks sound guides is that it introduces the concept of Constant-Q transforms and, admitting that there is currently no proprietary implementation of Constant-Q transforms, points to a workaround \cite{Klapuri}. It also highlights the Music Information Retrieval Evaluation eXchange (MIREX) page for updates on audio processing technologies \cite{mirexWiki}.

\paragraph{Graphics}
openFrameworks allows users to generate both 2D and 3D graphics. Their implementation comes in the form of classes that allow the user to define the shape or points of the graphic that they wish to generate. openFrameworks also provides an API that provides users with simplified functions accessing the more common functions of OpenGL. The documentation for the graphics classes seem more comprehensive than those of the sound classes.

\subsubsection{Creating a new project}
Creating a new project is straightforward: users simply use the \verb|projectGenerator| application. It does initially take awhile to compile an openFrameworks project, however. With the provided examples and online documentation, a simple animation that reacts to an incoming audio signal was created.\footnotemark The results of which are shown in Figure \ref{fig:OFVisualiser}.
\footnotetext{The source code can be accessed here: \url{https://github.com/jkmingwen/capstone/tree/master/Reports/Framework\%20Review/code/exploringSoundStream}.}


\begin{figure}
  \begin{subfigure}{0.5\textwidth}
    \includegraphics[width = 0.9\linewidth]{OF/simplesound_soft}
    \caption{Simple, soft sound}
  \end{subfigure} 
 \begin{subfigure}{0.5\textwidth}
    \includegraphics[width = 0.9\linewidth]{OF/simplesound_med}
    \caption{Simple, medium sound}
  \end{subfigure}
  \begin{subfigure}{0.5\textwidth}
    \includegraphics[width = 0.9\linewidth]{OF/complexsound_soft}
    \caption{Complex, soft sound}
  \end{subfigure}
  \begin{subfigure}{0.5\textwidth}
    \includegraphics[width = 0.9\linewidth]{OF/complexsound_perc}
    \caption{Complex, loud sound}
  \end{subfigure}
  \caption{Series of screenshots of visualiser implemented in openFrameworks}
  \label{fig:OFVisualiser}
\end{figure}

The graphics are generated by representing the various amplitudes of samples stored in a buffer. Samples 1 to 256 are represented by circles on an x-axis. Their displacement along the y-axis reflects their corresponding amplitudes. Finally, the overall volume of the buffer is mapped to the radius of the circles. This representation was surprisingly effective at conveying the timbre and volume of the audio input. Mapping the radius of the circles to the overall volume of the audio input also led to some interesting patterns that conveyed the complexity of an incoming signal by displaying an increasingly chaotic and indiscernable shape.

\subsubsection{Current implementations}
The following are some interesting projects where openFrameworks was used as a development framework:

\begin{enumerate}
\item fuse*'s \textit{MULTIVERSE} \cite{multiverse}
\item fuse*'s \textit{D\"okk} \cite{dokk}
\item graycode's \textit{\#include red} \cite{includered}
\item Ryoichi Kurokawa's \textit{node 5:5} \cite{KurokawaNode}
\end{enumerate}

\begin{figure}[h]
  \includegraphics[width = 0.9\linewidth]{OF/multiverse1}
  \caption{Screenshot of \textit{MULTIVERSE} \cite{multiverse}}
  \label{fig:multiverse1}
  \includegraphics[width = 0.9\linewidth]{OF/multiverse2}
  \caption{Screenshot of \textit{MULTIVERSE} \cite{multiverse}}
  \label{fig:multiverse2}
\end{figure}

A project of note is fuse*'s \textit{MULTIVERSE}. The visuals in \textit{MULTIVERSE} are generated through implementing particle behaviours in a simulated environment. The particles are programmed to react to each other, as well as to the space around them. Different behaviours are exhibited by altering the vector field that the particles are placed in. Shader programs were used to maximise hardware performance and to optimise the graphics pipline on the Graphics Processing Unit (GPU) \cite{multiverse}. As seen in Figures \ref{fig:multiverse1} and \ref{fig:multiverse2}, these behaviours can result in some particularly stunning visuals.  Kurokawa's \textit{node 5:5} had a similar aesthetic that was more closely tied to sound. It might be worth taking a look into how to best generate these graphics efficiently.

\subsubsection{Issues}
One of the main issues with openFrameworks is that projects have to be contained within the version it was implemented in. The documentation on their Git page states:

\begin{quote}
OF releases are designed to be self-contained. You can put them anywhere on your hard drive, but it's not possible to mix different releases of OF together, so please keep each release (0.8.0, 0.8.1) separate. Projects may generally work from release to release, but this is not guaranteed \cite{OFgithub}.
\end{quote}

This means that, after starting a project, running it on newer versions of openFrameworks will require careful version control. Finally, there were some issues with compiling openFrameworks on Xcode --- one of the Integrated Development Editors (IDEs) suggested --- but a quick look on the Git repository page provided a fix by editing the corresponding \verb|.xcconfig| file \cite{OFissue}.

\end{document}
%%% Local Variables:
%%% mode: latex
%%% TeX-master: "../../main_report1"
%%% End:
