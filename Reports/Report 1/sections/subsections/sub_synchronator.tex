\documentclass[../../main_report1.tex]{subfiles}

\begin{document}

\subsection{SYNCHRONATOR HD}
The \textit{SYNCHRONATOR HD} is a video synthesizer by Bas van Koolwijk and Gert-Jan Prins. The visuals of the \textit{SYNCHRONATOR HD} are generated by adding ``video sync pulses and color coding signals to [the] audio input...disguising the audio as a video signal'' \cite{VanKoolwijk}. It therefore seems to employ some form of analogue processing to convert an audio input into a video output. This results in visuals that appear to have a lot more gradation than those produced by the \textit{ETC}. Examples of the visuals are shown in Figure \ref{fig:synchronatorstills}. \par

\begin{figure}
  \begin{subfigure}{0.5\textwidth}
    \includegraphics[width = 0.9\linewidth]{Report1/synchronator_bw}
    \caption{B\&W visual from \textit{SYNCHRONATOR} \cite{VanKoolwijka}}
    \label{fig:synchronatorstillbw}
  \end{subfigure}
  \begin{subfigure}{0.5\textwidth}
    \includegraphics[width = 0.9\linewidth]{Report1/synchronator_col}
    \caption{Visual from green channel \cite{VanKoolwijka}}
    \label{fig:synchronatorstillcol}
  \end{subfigure}
  \caption{Stills from visuals produced by \textit{SYNCHRONATOR}}
  \label{fig:synchronatorstills}
\end{figure}

While users do have the option of using a separate device, called the \textit{SYNCHRONATOR ColorControl}, to mix two channels of audio into the RGB channels of the \textit{SYNCHRONATOR HD}, it is not designed to be played ``as an instrument'' in the way that the \textit{ETC} is designed to. The \textit{SYNCHRONATOR HD} and the \textit{SYNCHRONATOR ColorControl} are pictured in Figure \ref{fig:synchronatordevices}. \par
\begin{figure}
  \begin{subfigure}{0.5\textwidth}
    \includegraphics[width = 0.9\linewidth]{Report1/synchronatorhd}
    \caption{\textit{SYNCHRONATOR HD} \cite{VanKoolwijk}}
    \label{fig:synchronatorhd}
  \end{subfigure}
  \begin{subfigure}{0.5\textwidth}
    \includegraphics[width = 0.9\linewidth]{Report1/synchronatorcolmixer}
    \caption{\textit{SYNCHRONATOR ColorControl} \cite{VanKoolwijk}}
    \label{fig:synchronatorcol}
  \end{subfigure}
  \caption{Two \textit{SYNCHRONATOR} devices}
  \label{fig:synchronatordevices}
\end{figure}
\end{document}
The \textit{SYNCHRONATOR HD} is thus distinct from implementations that utilise digital processing to produce visuals. Nonetheless, it does provide valuable insight into a possible alternative method on how one might process an audio input to produce a video output. It also demonstrates an implementation of a video synthesizer that seems to encourage the user to focus on other aspects of their performance after deciding on a setting rather than having to constantly tweak parameters.
%%% Local Variables:
%%% mode: latex
%%% TeX-master: "../../main_report1"
%%% End:
