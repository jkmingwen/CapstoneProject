\documentclass[../main_report1.tex]{subfiles}

\begin{document}

\section{Video synthesizers} \label{sec:vsynths}
The review of existing video synthesizers was isolated to those implemented on some form of embedded system, furthermore, all are video synthesizers that take in an audio signal as input to produce a visual output. This was done to ensure that the scope of the review fell in line with the goals of the project. The focus when examining each product was how they generated visuals, as well as the means with which users were expected to use them. It ends with a section on where this project aims to sit in relation to the current range of video synthesizers.

\subfile{subsections/sub_etc}
\subfile{subsections/sub_synchronator}
\subfile{subsections/sub_opz}

\subsection{Relating project to current video synthesizers}
This project will attempt to amalgamate the implementations reviewed above. It aims to combine the flexibility of standalone video synthesizers like the \textit{ETC} and the \textit{SYNCHRONATOR HD} while attempting to produce graphics that match the visual gradations of the \textit{SYNCHRONATOR HD} and, to a certain extent, the complexity afforded by the \textit{OP-Z}. It will be designed to be used in a similar fashion to the \textit{SYNCHRONATOR HD}, where the user is encouraged to focus on creating music and observing its effect on the video output after fine-tuning the look and responsiveness of the video synthesizer. Having established the relation of video synthesizers to the field of audiovisual expression, as well as the relation of this project to the current technologies available, we move on to a review of the various coding frameworks and boards that this project could be implemented on.

\end{document}
%%% Local Variables:
%%% mode: latex
%%% TeX-master: "../main_report1"
%%% End: