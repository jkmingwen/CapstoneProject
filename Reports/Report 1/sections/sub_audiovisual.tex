\documentclass[../main_report1.tex]{subfiles}

\begin{document}
\section{Audiovisual Expression}
There are two animators whose works emphasise the relations between music and visual expression -- Norman McLaren and John Whitney. McLaren uses the term ``animated sound'' to explain how --- by painting in the audio tracks of film strips --- one is able to synthesize audio \cite{McLaren1953}. In documenting his methods of synthesizing sound, McLaren mapped the shapes of his drawings to the qualities of the sounds they produced \cite{McLaren1953}. Whitney, on the other hand, attempts to understand how to recreate the tension and release of music through visuals. In his book, \textit{Digital Harmony}, Whitney draws parallels between music and structured visual movements \cite{Whitney1980}. He envisions ``a visual world of harmony to which there must be innate human responses, just as in the world of music'' --- arguing that certain forms of motion elicit emotional responses in the same way that harmony and disharmony in music does \cite{Whitney1980}. About a decade later, when advances in technology allowed for algorithmically designed graphics-generating capabilities on personal computers, Whitney restated his argument that the parallels between music and certain visual patterns can come together to create something greater than its parts \cite{Whitney1991}. Crucially, Whitney believes that ``only \textbf{structured} motion begets emotion'' \cite{Whitney1980} and that the architecture of music is embodied by algorithmic resonance and ratio \cite{Whitney1991}. These are operations that computers are particularly adept at performing. \par

The works of McLaren and Whitney emphasise the links between the audio and visual realms. It is upon these ideas that video synthesizers can be understood. The current technologies of audiovisual expression, in the form of video synthesizers explored in Section \ref{sec:vsynths}, could thus be seen as a means to realise the ideas propagated by McLaren and Whitney. There can be a meaningful link between music and visuals and this thought provides an impetus to the creation of video synthesizers made specifically to accompany an audio input.

\end{document}
%%% Local Variables:
%%% mode: latex
%%% TeX-master: "../main_report1"
%%% End:
