\documentclass[../../main_report2.tex]{subfiles}

\begin{document}

\subsection{SYNCHRONATOR HD}
The \textit{SYNCHRONATOR HD} is a video synthesizer by Bas van Koolwijk and Gert-Jan Prins. The visuals of the \textit{SYNCHRONATOR HD} are generated by adding ``video sync pulses and color coding signals to [the] audio input...disguising the audio as a video signal'' \cite{VanKoolwijk}. It seems to employ some form of analogue processing to convert an audio input into a video output. This results in visuals that appear to have a lot more gradation than those produced by the \textit{ETC}. Examples of the visuals are shown in Figure \ref{fig:synchronatorstills}. \par

\begin{figure}
  \begin{subfigure}{0.5\textwidth}
    \includegraphics[width = 0.9\linewidth]{Report2/synchronator_bw}
    \caption{B\&W visual \cite{VanKoolwijka}}
    \label{fig:synchronatorstillbw}
  \end{subfigure}
  \begin{subfigure}{0.5\textwidth}
    \includegraphics[width = 0.9\linewidth]{Report2/synchronator_col}
    \caption{Visual from green channel \cite{VanKoolwijka}}
    \label{fig:synchronatorstillcol}
  \end{subfigure}
  \caption{Stills from visuals produced by \textit{SYNCHRONATOR}}
  \label{fig:synchronatorstills}
\end{figure}

While users do have the option of using a separate device, called the \textit{SYNCHRONATOR ColorControl}, to mix two channels of audio into the RGB channels of the \textit{SYNCHRONATOR HD}, it is not designed to be played ``as an instrument'' in the way that the \textit{ETC} is designed to. The \textit{SYNCHRONATOR HD} and the \textit{SYNCHRONATOR ColorControl} are pictured in Figure \ref{fig:synchronatordevices}. \par
\begin{figure}
  \begin{subfigure}{0.5\textwidth}
    \includegraphics[width = 0.9\linewidth]{Report2/synchronatorhd}
    \caption{\textit{SYNCHRONATOR HD} \cite{VanKoolwijk}}
    \label{fig:synchronatorhd}
  \end{subfigure}
  \begin{subfigure}{0.5\textwidth}
    \includegraphics[width = 0.9\linewidth]{Report2/synchronatorcolmixer}
    \caption{\textit{SYNCHRONATOR ColorControl} \cite{VanKoolwijk}}
    \label{fig:synchronatorcol}
  \end{subfigure}
  \caption{Two \textit{SYNCHRONATOR} devices}
  \label{fig:synchronatordevices}
\end{figure}
\end{document}
The \textit{SYNCHRONATOR HD} is thus distinct from implementations that utilise digital processing to produce visuals. While using analogue signal processing likely means that latency is less of an issue in comparison to digital signal processing, it also means that the \textit{SYNCHRONATOR HD} is likely less flexible in terms of its visual output as well as the parameters of the audio signal it can track. Nonetheless, it does provide valuable insight into a possible alternative method on how one might process an audio input to produce a video output. It also demonstrates an implementation of a video synthesizer that seems to encourage the user to focus on other aspects of their performance after deciding on a setting rather than having to constantly tweak parameters.
%%% Local Variables:
%%% mode: latex
%%% TeX-master: "../../main_report2"
%%% End:
